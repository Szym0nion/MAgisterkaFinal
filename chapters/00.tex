\subsubsection*{Tytuł pracy} 
\Title

\subsubsection*{Streszczenie}  
Praca magisterska poświęcona jest zagadnieniom bezpieczeństwa i ochrony prywatności w systemach Internetu Rzeczy (IoT). Wraz z dynamicznym rozwojem technologii IoT rośnie znaczenie skutecznych mechanizmów zabezpieczających, które muszą jednocześnie chronić dane użytkowników i uwzględniać ograniczenia zasobowe urządzeń.

W pracy przeanalizowano kluczowe wyzwania związane z bezpieczeństwem w IoT, w tym zagrożenia takie jak nieautoryzowany dostęp, ataki typu DDoS czy naruszenia prywatności. Omówiono stosowane obecnie rozwiązania techniczne, w tym metody szyfrowania, protokoły komunikacyjne i systemy uwierzytelniania, zwracając szczególną uwagę na ich wpływ na wydajność systemów IoT.

Głównym celem badania było znalezienie optymalnej równowagi między poziomem bezpieczeństwa, ochroną prywatności użytkowników a efektywnym wykorzystaniem zasobów urządzeń IoT. W pracy zaproponowano szereg rozwiązań optymalizacyjnych, które mogą przyczynić się do poprawy bezpieczeństwa systemów IoT bez nadmiernego obciążania ich ograniczonych możliwości obliczeniowych.

Wyniki przeprowadzonych analiz mogą stanowić wartościowy punkt wyjścia dla producentów urządzeń IoT oraz dostawców rozwiązań chmurowych, pomagając w projektowaniu bardziej bezpiecznych i przyjaznych dla użytkownika systemów Internetu Rzeczy. Praca wskazuje także kierunki dalszych badań w tej dynamicznie rozwijającej się dziedzinie.

\subsubsection*{Słowa kluczowe} 
Internet Rzeczy (IoT), bezpieczeństwo cybernetyczne, ochrona prywatności, szyfrowanie danych, optymalizacja zabezpieczeń
\newpage
\subsubsection*{Thesis title} 
\begin{otherlanguage}{british}
\TitleAlt
\end{otherlanguage}

\subsubsection*{Abstract} 
\begin{otherlanguage}{british}
This master's thesis focuses on security and privacy issues in Internet of Things (IoT) systems. As IoT technology rapidly develops, there is a growing need for effective security mechanisms that can protect user data while accounting for device resource limitations.

The study examines key IoT security challenges, including threats such as unauthorized access, DDoS attacks, and privacy breaches. It analyzes current technical solutions, including encryption methods, communication protocols, and authentication systems, with particular attention to their impact on IoT system performance.

The main research objective was to find an optimal balance between security levels, user privacy protection, and efficient utilization of IoT device resources. The thesis proposes several optimization approaches that could enhance IoT security without excessively straining the limited computational capabilities of these devices.

The analysis results provide valuable insights for IoT device manufacturers and cloud solution providers, supporting the development of more secure and user-friendly IoT systems. The study also identifies promising directions for future research in this rapidly evolving field.
\end{otherlanguage}
\subsubsection*{Key words}  
\begin{otherlanguage}{british}
Internet of Things (IoT), cybersecurity, privacy protection, data encryption, security optimization
\end{otherlanguage}

